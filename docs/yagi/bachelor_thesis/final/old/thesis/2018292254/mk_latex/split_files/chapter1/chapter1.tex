\chapter{はじめに}\label{ux306fux3058ux3081ux306b}

\section{目的}\label{ux76eeux7684}

西谷研究室で使われているユーザメモソフト,my
helpの振る舞いを制御しているサブコマンドは,マイナスを付した省略記法が取られている.プログラミング初心者にとってこの省略記法は,覚えにくかったりわかりにくかったりするという問題があり,現在はフルワードを使った自然言語に近い記述法が多く用いられている.
そこで,本研究ではコマンドラインツール作成ライブラリを自然言語に近いサブコマンド体系を実装しやすいライブラリであるThorに変更する.rubyの標準ライブラリであるoptparseで作成されているmy
helpをThorによって書き直し,異なった2つライブラリで作成されたmy
helpの使用感を比較検討することを目的とする.

    