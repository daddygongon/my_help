% Created 2018-09-19 水 22:44
\documentclass[11pt]{article}
\usepackage[utf8]{inputenc}
\usepackage[T1]{fontenc}
\usepackage{fixltx2e}
\usepackage{graphicx}
\usepackage{longtable}
\usepackage{float}
\usepackage{wrapfig}
\usepackage{rotating}
\usepackage[normalem]{ulem}
\usepackage{amsmath}
\usepackage{textcomp}
\usepackage{marvosym}
\usepackage{wasysym}
\usepackage{amssymb}
\usepackage{hyperref}
\tolerance=1000
\author{Yamada Tomoko}
\date{\today}
\title{question}
\hypersetup{
  pdfkeywords={},
  pdfsubject={},
  pdfcreator={Emacs 25.3.1 (Org mode 8.2.10)}}
\begin{document}

\maketitle
\tableofcontents


\section{質問内容}
\label{sec-1}

\subsection{pdfの問題}
\label{sec-1-1}
open README.pdfではadobeで開いてしまうため,
open . をして,README.texで直接いじっているけど
実際は良くないんじゃないか.

\begin{description}
\item[{回答}] acrobatだったらいいですよ.open README.texでtexshopが起動するんでは?
\end{description}
apple-iでinformation開いてこのapplicationで開くを設定すると指定したアプリで開くようになります.
この辺りは,わからんかったら聞けば答えるよ.
でも,ネットでグーぐればいくらでも出てくるので,自分でこうと思うことをやってみるのも大事.

\section{中間審査}
\label{sec-2}
\subsection{1}
\label{sec-2-1}
知識の習得についてもっと書く.my$_{\text{help}}$->AM/PMの話に飛ぶのは,ちょっと飛躍してる.
\subsection{2}
\label{sec-2-2}
何がしたいか分からない.

\subsection{3}
\label{sec-2-3}
unixでemacsのhelpが出るけど?
別にmy$_{\text{helpじゃなくてもすぐ出すことできるよね?}}$

\subsection{4}
\label{sec-2-4}
スキルレベル評価は誰が行うのか?私がいなくなった時どうしてしまうの?
\begin{itemize}
\item 評価の一般化を行う
\end{itemize}
\subsection{5}
\label{sec-2-5}
本当にwebで表示とか実現可能なのか
\subsection{6}
\label{sec-2-6}
wikiじゃダメなのか

\section{考え}
\label{sec-3}
\subsection{1}
\label{sec-3-1}
知識を他人と出し合うことによって,間違った知識の修正,知識の確立ができる.
my$_{\text{helpはmanの軽い版のhelpを目的に作られたけど,目標はそこじゃない.}}$
webで表示になると,wikiと同じようになる.
\begin{itemize}
\item これは,webで見れるけど,gitを使ってrepositoryにあるorgファイルを自分のパソコンにもpullして持ってこれるようにする.そうして,terminal上での参照,変更ができるようにする.
\end{itemize}

\subsection{2}
\label{sec-3-2}
前回の説明では,自分のオリジナルのmemoと言ってしまったが,人に伝えることも意識した書き方をしていく.

\subsection{3}
\label{sec-3-3}
これはちょっと置いとく

\subsection{4}
\label{sec-3-4}
評価の方法.
\begin{itemize}
\item レベル0
\end{itemize}
コマンドなど

\begin{itemize}
\item レベル1
\end{itemize}
コードにおける雛形の説明

とか??そんな感じ?

\subsection{5}
\label{sec-3-5}
ちゃんと構想ねる.

\subsection{6}
\label{sec-3-6}
wikiでもなあ
自分のパソコンに落とし込めれるんかな?
% Emacs 25.3.1 (Org mode 8.2.10)
\end{document}
