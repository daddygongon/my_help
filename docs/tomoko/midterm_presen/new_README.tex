\documentclass[a4j,twocolumn,uplatex]{jsarticle}
\usepackage[dvipdfmx]{graphicx}
\usepackage{url}
\usepackage{comment}

\setlength{\textheight}{275mm}
\headheight 5mm
\topmargin -30mm
\textwidth 185mm
\oddsidemargin -15mm
\evensidemargin -15mm
\pagestyle{empty}

\usepackage{listings,jlisting}
\lstset{%
  language={Ruby},%
  basicstyle={\small},%
  identifierstyle={\small},%
  commentstyle={\small\itshape},%
  keywordstyle={\small\bfseries},%
  ndkeywordstyle={\small},%
  stringstyle={\small\ttfamily},
  frame={tb},
  breaklines=true,
  columns=[l]{fullflexible},%
  numbers=left,%
  xrightmargin=0zw,%
  xleftmargin=3zw,%
  numberstyle={\scriptsize},%
  stepnumber=1,
  numbersep=1zw,%
  lineskip=-0.5ex,%
  showstringspaces=false%
}

\begin{document}

\title{memoソフトmy\_helpの改善}
\author{情報科学科 \hspace{5mm} 27014520 \hspace{5mm}山田智子}
\date{}
\maketitle

\section{背景}
私たちは,何か知識を得たときに度々メモを取る.
そのメモには自分自身にとって知識を思い出すために必要なキーワードが書いてある. 
しかし,そのメモを紛失してしまったり,
どこへやったかわからなくなってしまったりする.
そこで,メモを探す手間や紛失する可能性を無くすために
my\_helpというmemoソフトが開発されている.\cite{my_help}.

現在,my\_helpは自身の知識をmemoとして残すことができるシステムである.
そのmemoは自分にとって重要なものだが,
他人が知識を得ることにも重要な役割を担うと考えた.
そこで,より効率的な知識の習得方法として,AM/PMという考え方がある.

\begin{description}
\item[AM(acquisition metaphor)] 旧来の学習感.                                       
学習目標は個々を豊かにする, 学習とは何かを獲得する(acquisition)ことであり,
知るとは持つ,所有することである.
学習という行為が非常に個人レベルに押し込められた感じがある.
しかし,現在の社会では個人の能力が測られるという意味で, 
知識を所有することが不可欠である.

\item[PM(participation metaphor)] 新しい学習感.
学習あるいは学習者とは参加者であり,
テキストや教授者から知識を得るのではなく,
自らも参加者になって知識を共有する.
学会活動も学習の一部と考える.研究者が学会で認められるということが,
その分野での用語を使って参加者とコミュニケーションを取れることであり,
論文集を出すことや初心者向けのテキストを書いたりする活動も学習支援のひとつ.
 \end{description}
 
自分で作成したmy\_helpを公開することによって,
\begin{itemize}
\item 自分の間違いが修正される.
\item 人に教えることによって,知識の定着が促進される.
\end{itemize}
ことが期待される.

\section{my\_helpについて}
my\_helpはemacsのMarkdownであるorg-modeを利用したソフトなので,
org-modeのexport機能を利用すればHTMLやLaTeXなど様々なフォーマットに
変換可能である\cite{org-mode}.

org-modeで作成した文章はemacs以外でも利用できる.
例えば,githubでは.mdと同じ様に.orgに対応している.

\section{手法}
my\_helpで作成されたmemoの共有方法を示す.
\subsection{リモートコンピュータと同期}
\label{sec-2-1}
リモートコンピュータに各自アカウントを作成する.
公開リポジトリour\_helpをgit cloneして,
その中のmembersに各自のディレクトリを作成する.                                                                  

その各自のディレクトリ内にmy\_helpで作成したmemoを蓄積する.                                          

\subsection{my\_helpのmemoを集める}
\label{sec-2-2}
our\_help/member/USER\_NAMEにpushされたorgファイルのリンクを
our\_help/savings/web.orgに記述する.
HTMLでも閲覧できる.

\subsection{memoを見る}
\label{sec-2-3}
web.org上にあるファイルを閲覧する.

\subsection{欲しいファイルを選択する}
\label{sec-2-4}
web.org上にある欲しいファイルを選択する.

\subsection{選択したファイルをpullする}
\label{sec-2-3}
選択したファイルをGithub からpullする.


\begin{comment}
%{\small\setlength\baselineskip{10pt}	% 参考文献は小さめの文字で行間を詰めてある
\begin{verbatim}
> my_Help list emacs -c
- emacsのキーバインド
- 
特殊キー操作
-   C-f, controlキーを押しながら    'f'
-   M-f, escキーを押した後一度離して'f'
-     操作の中断C-g, 操作の取り消し(Undo) C-x u
-----
cursor
- C-f, move Forwrard,    前or右へ
- C-b, move Backwrard,   後or左へ
...
\end{verbatim}


my\_helpはshell上のdirectory位置によらずどこからでも
呼び出せて,applicationを切り替えることなく参照できる.
また,編集が手軽にできることから
自分独自のメモを取ることによって,
記憶の定着を促すツールとしての活用を意図している.
% 補助脳,
% 検索ができるようにする
\end{comment}


\section{開発目標}
本研究では,my\_helpを改善し,
発展させて行くことが今後の目標である.
具体的には,
\begin{enumerate}
\item my\_helpのsearch機能追加
\item スキルレベルによる難易度表示
\item point付加によるranking
\end{enumerate}
などを取り入れた共有システムを作成する.

{\small\setlength\baselineskip{10pt}	% 参考文献は小さめの文字で行間を詰めてある
\begin{thebibliography}{9}
\bibitem{my_help}\url{https://github.com/daddygongon/my_help}, Daddygongon, (18/09/16 accessed) .
\bibitem{org-mode} \url{https://qiita.com/dwarfJP/items/594a8d4b0ac6d248d1e4}, (18/09/16 accessed).
\bibitem{sfard}“On Two Metaphors for Learning and the Dangers of Choosing Just one”, Anna
Sfard, Educational Researcher, 27(1998), 4–13.
\end{thebibliography}
}
% Emacs 25.3.1 (Org mode 8.2.10)
\end{document}
