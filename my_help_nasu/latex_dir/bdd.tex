\section{BDD}
ビヘイビア駆動開発(Behaviour-Driven Development : BDD)は,テスト駆動開発(Test-Driven Development : TDD)の工程への理解を深め,それをうまく説明しようとして始まりました.

BDDは構造ではなく振る舞いに焦点を合わせます.
それは開発のすべてのレベルでいっかんしてそうなります.
2つの都市の間の距離を計算するオブジェクトのことであっても,サードパーティのサービスに検索を委任する別のオブジェクトのことであっても,
あるいはユーザーが無効なデータを入力したときにフィードバックを提供する別の画面であっても,それはすべて振る舞いなのです.
これを飲み込んでしまえば,コードに取り組むときの考え方が変わります.
オブジェクトの構造よりも,ユーザーとシステムの間でのやり取り,つまりオブジェクトの間でのやり取りについて考えるようになります.

ソフトウェア開発チームが直面する問題のほとんどは,コミュニケーションの問題であると考えています.
BDDの目的は,ソフトウェアが使われる状況を説明するための言語を単純かすることで,コミュニケーションを後押しすることです.
つまり,あるコンテキストで(Given),あるイベントが発生すると(When),ある結果が期待されます(Then).
BDDにおけるGiven, When,Thenの3つの単語は,アプリケーションやオブジェクトを,それらの振る舞いに関係なく表現するために使われる単純な単語です.
ビジネスアナリスト,テスト担当者,開発者は皆,それらをすぐに理解します.
これらの単語はCucumberの言語に直接埋め込まれています[1, pp.3-6].

BDDサイクルの図を以下に示します[1, 9pp.].
\verb|{{attach_view(my_help_nasu.001.jpg)}}|
\verb|{{attach_view(my_help_nasu1.001.jpg)}}|

