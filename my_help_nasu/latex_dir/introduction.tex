\section{序論}
プログラム開発では,統合開発環境がいくつも用意されているが,多くの現場では,terminal上での開発が一般的です.
ところが,プログラミング初心者はterminal上でのcharacter user interface(CUI)を苦手としています.
プログラミングのレベルが上がるに従って,
shell commandやfile directory操作, process制御にCUIを使うことが常識です.

この不可欠なCUIスキルの習得を助けるソフトとして,ユーザメモソフトmy\_helpがruby gemsに置かれています.
このcommand line interface(CLI)で動作するソフトは,helpをterminal上で簡単に提示するものです.
また,初心者が自ら編集することによって,すぐに参照できるメモとしての機能を提供しています.
これによって,terminal上でちょっとした調べ物ができるため,作業や思考が中断することなく
プログラム開発に集中できることが期待でき,初心者のスキル習得が加速することが期待できます.

しかし,Ruby gemsとして提供されているこのソフトは,動作はするがテストが用意されていません.
慣れた開発者は,テストを見ることで仕様を理解するのが常識です.
今後ソフトを進化させるために共同開発を進めていくには,仕様や動作の標準となるテスト記述が不可欠となります.

そこで,本研究では,ユーザメモソフトであるmy\_helpのテストを開発することを目的とします.
本研究では、テスト駆動開発の中でも,ソフトの振る舞いを記述します.
Behavior Driven Development(BDD)に基づいてテストを記述していきます.
Rubyにおいて、BDD環境を提供する標準的なフレームワークであるCucumberとRSpecを用いて,
my\_helpがどのような振る舞いをするのかを記述します.
Cucumberは自然言語で振る舞いを記述することができるため,ユーザにとって,わかりやすく振る舞いを確認することができます.

