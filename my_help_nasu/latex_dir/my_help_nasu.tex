\documentclass[12pt,a4paper]{jsarticle}
\usepackage[dvipdfmx]{graphicx}
\usepackage[dvipdfmx]{color}
\usepackage{listings,jlisting}% to use japanese correctly, install jlistings.
\lstset{
  basicstyle={\small\ttfamily},
  identifierstyle={\small},
  commentstyle={\small\itshape\color{red}},
  keywordstyle={\small\bfseries\color{cyan}},
  ndkeywordstyle={\small},
  stringstyle={\small\color{blue}},
  frame={tb},
  breaklines=true,
  numbers=left,
  numberstyle={\scriptsize},
  stepnumber=1,
  numbersep=1zw,
  xrightmargin=0zw,
  xleftmargin=3zw,
  lineskip=-0.5ex
}
\lstdefinestyle{customCsh}{
  language={csh},
  numbers=none,
}
\lstdefinestyle{customRuby}{
  language={ruby},
  numbers=left,
}
\lstdefinestyle{customTex}{
  language={tex},
  numbers=none,
}
\lstdefinestyle{customJava}{
  language={java},
  numbers=left,
}
\begin{document}
\title{卒業論文\\
\vspace{4cm} ユーザメモソフトmy\_helpの開発}
\author{ 関西学院大学 理工学部 情報科学科\\\\2535 那須比呂貴}
\date{\vspace{3cm} 2017年  3月\\
\vspace{3cm} 指導教員  西谷 滋人 教授}
\maketitle
\setcounter{tocdepth}{4}
\tableofcontents

\tableofcontents
プログラム開発では,統合開発環境がいくつも用意されているが,多くの現場では,terminal上での開発が一般的です.
ところが,プログラミング初心者はterminal上でのcharacter user interface(CUI)を苦手としています.
この不可欠なCUIスキルの習得を助けるソフトとして,ユーザメモソフトmy\_helpがruby gemsに置かれています.
このcommand line interface(CLI)で動作するmy\_helpは,helpをterminal上で簡単に提示するものです.
また,初心者が自ら編集することによって,すぐに参照できるメモとしての機能を提供しています.
これにより,プログラム開発に集中できることが期待でき,初心者のスキル習得が加速することが期待できます.
しかし,Ruby gemsとして提供されているこのソフトは,動作はするがテストが用意されていません.
今後ソフトを進化させるために共同開発を進めていくには,仕様や動作の標準となるテスト記述が不可欠となります.
そこで,本研究では,ユーザメモソフトであるmy\_helpのテストを開発することを目的とします.
本研究では、テスト駆動開発の中でも,ソフトの振る舞いを記述します
Behavior Driven Development(BDD)に基づいてテストを記述していきます.
Rubyにおいて、BDD環境を提供する標準的なフレームワークであるCucumberとRSpecを用いて,
my\_helpがどのような振る舞いをするのかを記述します.
Cucumberは自然言語で振る舞いを記述することができるため,ユーザにとって,わかりやすく振る舞いを確認することができます.
ここでは,実際にBDDの流れにそって,my\_helpのある一例を取り上げてのテスト開発を進めていきます.
また,my\_helpの頻繁に使うコマンドの振る舞いを記述しています.
このことにより,来年の本研究室の研究生のスキルがより向上することが期待されると考えています.


\section{序論}
プログラム開発では,統合開発環境がいくつも用意されているが,多くの現場では,terminal上での開発が一般的です.
ところが,プログラミング初心者はterminal上でのcharacter user interface(CUI)を苦手としています.
プログラミングのレベルが上がるに従って,
shell commandやfile directory操作, process制御にCUIを使うことが常識です.

この不可欠なCUIスキルの習得を助けるソフトとして,ユーザメモソフトmy\_helpがruby gemsに置かれています.
このcommand line interface(CLI)で動作するソフトは,helpをterminal上で簡単に提示するものです.
また,初心者が自ら編集することによって,すぐに参照できるメモとしての機能を提供しています.
これによって,terminal上でちょっとした調べ物ができるため,作業や思考が中断することなく
プログラム開発に集中できることが期待でき,初心者のスキル習得が加速することが期待できます.

しかし,Ruby gemsとして提供されているこのソフトは,動作はするがテストが用意されていません.
慣れた開発者は,テストを見ることで仕様を理解するのが常識です.
今後ソフトを進化させるために共同開発を進めていくには,仕様や動作の標準となるテスト記述が不可欠となります.

そこで,本研究では,ユーザメモソフトであるmy\_helpのテストを開発することを目的とします.
本研究では、テスト駆動開発の中でも,ソフトの振る舞いを記述します.
Behavior Driven Development(BDD)に基づいてテストを記述していきます.
Rubyにおいて、BDD環境を提供する標準的なフレームワークであるCucumberとRSpecを用いて,
my\_helpがどのような振る舞いをするのかを記述します.
Cucumberは自然言語で振る舞いを記述することができるため,ユーザにとって,わかりやすく振る舞いを確認することができます.


\section{先行研究,方法}
ここでは、本研究で使用するcucumberの特徴について詳述する.
cucumberはビヘイビア駆動開発(bdd)を実現するフレームワークである.
まずはbddの現れた背景や現状を示した後、Cucumberの記述の具体例を示して,
その特徴を詳述する.
さらに,本研究の対象となるmy\_helpの振る舞いを使用法とともに示す.


\subsection{RSpecとBDDについて}
ビヘイビア駆動開発(Behaviour-Driven Development : BDD)は,テスト駆動開発(Test-Driven Development : TDD)の工程への理解を深め,それをうまく説明しようとして始まりました.TDDの持つ単語のイメージが構造のテストを中心とするべしというのに対して,BDDはソフトの振る舞いに中心をおきなさいという意図があります.この違いが,初めに考えるべきテストの性質を変化させ,構造ではなく振る舞いを中心にテストを構築するという意識をもたせてくれます.

さらに,ソフトの中で,オブジェクト同士がコミュニケーションをとるように,実世界において開発チームやテストチーム,あるいはドキュメントチーム間のコミュニケーションの取り方をシステムで提供しようというのがBDDのフレームワークです.CucumberとRSpecはこれを実現する一つのシステムとして提供されています.

RSpecとCucumberの関係を図に示しました.これは,RSpec本から書き写した図です[1, pp.9].RSpecでテストを書くと一つ一つのfunctionあるいはmethodレベルでRed, Green, Refactoringを行うべしという意図があります.一方で,もっと大きな枠組み,つまりシステムレベルでもこれらのステップは必要です.ところが,それをRSpecで書くのには無理があります.このレベルのテスト記述をしやすくするのが,Cucumberです.そこでもRed, Green, Refactoringが必要で,そこでサイクルが回ることを意図しています.

\begin{figure}[htbp]\begin{center}
\includegraphics[width=10cm,bb= 0 0 737 553]{../figs/./my_help_nasu.png}
\caption{RSpecとCucumberのRed-Green-Refactoringサイクル間の関係.}
\label{default}\end{center}\end{figure}
BDDの基本的な考え方は次の通りまとめられています.

\begin{quotation}
BDDの目的は,ソフトウェアが使われる状況を説明するための言語を単純化することで,ソフトウェア開発チームのコミュニケーションを後押しすることです.つまり,あるコンテキストで(Given),あるイベントが発生すると(When),ある結果が期待されます(Then).BDDにおけるGiven, When, Thenの3つの単語は,アプリケーションやオブジェクトを,それらの振る舞いに関係なく表現するために使われる単純な単語です.ビジネスアナリスト,テスト担当者,開発者は皆,それらをすぐに理解します.これらの単語はCucumberの言語に直接埋め込まれています[1, pp.3-6].

\end{quotation}
手順を書き直すと次の通りです.

\begin{figure}[htbp]\begin{center}
\includegraphics[width=10cm,bb= 0 0 737 553]{../figs/./my_help_nasu1.001.jpg}
\caption{RSpecとCucumberの手順.}
\label{default}\end{center}\end{figure}
まずCucumberで一つのシナリオに焦点を当てて,その振る舞いを記述するfeatureを書きます.一つずつつぶしていくのがこつです.一つのfeatureが書けたら、次に,それぞれfeatureを実現するステップに分けて仕様を決めて行きます.これはTDDのred green refactoringの前に行う作業,「仕様をきめる」に対応しています.このプロセスが終了したら,RSpecに行きます.
RSpecでは実際にテストコードを書き,ここでもred, green, refactoringを行います.
RSpecが成功したら,Cucumberのrefactoringを行います.



\subsection{Cucumberについて}
\subsubsection{概要}
Cucumberが提供するBDDの内容をまとめると

\begin{quotation}
BDDはフルスタックのアジャイル開発技法です.BDDはATDP(Acceptance Test-Driven Planning)と呼ばれるAcceptance TDDの一種を含め,エクストリームプログラミングからヒントを得ています.ATDPでは,顧客受け入れテストを導入し,それを主体にコードの開発を進めて行きます.それらは顧客と開発チームによる共同作業の結果であることが理想的です.開発チームによってテストが書かれた後,顧客がレビューと承認を行うこともあります.いずれにしても,それらのテストは顧客と向き合うものなので,顧客が理解できる言語とフォーマットで表現されていなければなりません.Cucumberを利用すれば,そのための言語とフォーマットを手に入れることができます.Cucumberは,アプリケーションの機能とサンプルシナリオを説明するテキストを読み取り,そのシナリオの手順に従って開発中のコードとのやり取りを自動化します[1, 7pp.].

\end{quotation}
と記されている.

\subsubsection{features}
Cucumberでは先の引用にある通り、振る舞いをシナリオとしてまず記述します.
次に,英語のfeaturesのひな形を示します.
\begin{quote}\begin{verbatim}
% cat ./featrues/sample_e.feature
Feature: Description of feature

Scenario: Description of scenario
  Given I want to explain scenario
  Then I investigate
  When I know the meaning
\end{verbatim}\end{quote}
ファイルの先頭で,
\begin{quote}\begin{verbatim}
# language: ja
\end{verbatim}\end{quote}
と記すと日本語でのkeywordが認識されます.下記にmy\_todoに対するfeaturesファイルの具体例を示す.
\begin{quote}\begin{verbatim}
# language: ja

機能: todoの更新を行う
todoは更新していくものであり,新しく書いたり終わったものを消したいので
バックアップをとって,過去のtodoを残しておく

シナリオ: コマンドを入力してtodoを更新していく
 前提 todoを編集したい
 もし "my_todo --edit"と入力する
 ならば editが開かれる
 かつ 自分のtodoを書き込む

シナリオ: コマンドを入力してバックアップをとる
 前提 todoの編集が終わった
 もし "my_todo --store [item]"と入力する
 ならば itemのバックアップを取る
\end{verbatim}\end{quote}
このようにFeature, Scenario, Given, Then, Whenなどのcucumberが解釈する大文字で始まる
keywordsに対して,それぞれ機能,シナリオ,前提,もし,ならばなどの単語があてられています.
この機能により,より自然な日本語でfeaturesを書くことができ,
顧客にもわかりやすく,開発者も書きやすくなっています。

featureファイルで用意されているkeywordは
\begin{quote}\begin{verbatim}
cucumber --i18n LANG
\end{verbatim}\end{quote}
によって表示される.LANG=ja, enに対しては下記の通りになっています.

\begin{table}[htbp]\begin{center}
\caption{}
\begin{tabular}{llll}
\hline
keyword   &ja(japanese)   &en(english)  \\ \hline
feature  &"フィーチャ", "機能"      &"Feature", "Business Need", "Ability"     \\
background  &"背景"  &"Background"   \\
scenario  &"シナリオ"        &"Scenario"     \\
scenario\_outline   &"シナリオアウトライン", "シナリオテンプレート","テンプレ", "シナリオテンプレ"   &"Scenario Outline", "Scenario Template"   \\
examples  &"例", "サンプル"   &"Examples", "Scenarios"          \\
given   &"* ", "前提"        &"* ", "Given "          \\
when   &"* ", "もし"        &"* ", "When "  \\
then   &"* ", "ならば"       &"* ", "Then "  \\
and    &"* ", "かつ"        &"* ", "And "   \\
but    &"* ", "しかし", "但し", "ただし"  &"* ", "But "   \\
given (code)   &"前提"   &"Given"        \\
when (code)   &"もし"     &"When"         \\
then (code)   &"ならば"   &"Then"         \\
and (code)    &"かつ"     &"And"          \\
but (code)    &"しかし", "但し","ただし"   &"But"          \\
\hline
\end{tabular}
\label{default}
\end{center}\end{table}
%for inserting separate lines, use \hline, \cline{2-3} etc.

\subsubsection{Cucumber,RSpecインストール}
まずrspecをgemでinstallする.

\begin{enumerate}
\item gem install rspec --version 2.0.0
\item rspec --help
\end{enumerate}
と入力して
\begin{quote}\begin{verbatim}
/Users/nasubi/nasu% rspec --help
Usage: rspec [options] [files or directories]
\end{verbatim}\end{quote}
のような表示がされていればinstallができている.次に,cucumberをinstallする

\begin{enumerate}
\item gem install cucumber --version 0.9.2
\item cucumber --help
\end{enumerate}
と入力して
\begin{quote}\begin{verbatim}
cucumber --help
Usage: cucumber [options] [ [FILE|DIR|URL][:LINE[:LINE]*] ]+
\end{verbatim}\end{quote}
のような表示がされていればinstallできている.

\subsubsection{ディレクトリー構造と使用手順}
cucumberはRubygemsの提供する基本directory構造での作業を前提としています.
その構造を表示すると次のようになります.
\begin{lstlisting}[style=]
bob% tree .
.
├── Gemfile
├── Rakefile
├── features
│   ├── hogehoge.feature
│   ├── step_definitions
│   │   └── hogehoge_step.rb
│   ├── support
│       ├── env.rb
├── lib
│   ├── daddygongon
│   │   ├── emacs_help.yml
│   │   ├── my_todo.yml
├── pkg
├── spec
│   ├── my_help_spec.rb
│   ├── my_todo
│   │   ├── todo_spec.rb
│   ├── spec_helper.rb
│   └── support
│       └── aruba.rb
\end{lstlisting}
カレントディレクトリ(.)の中にfeaturesというサブディレクトリを作成します.
そのfeaturesの中に書きたいシナリオを書いた,hogehoge.featureを作成します.
featureの具体例は上記に示してします.

次にシェルを開いて,カレントディレクトリで,
\begin{quote}\begin{verbatim}
cucumber features hogehoge.feature
\end{verbatim}\end{quote}
と入力します.そうすると以下のような出力が得られます.
\begin{lstlisting}[style=]
Feature: Description of feature

  Scenario: Description of scenario  # features/hogehoge.feature:3
    Given I want to explain scenario # features/hogehoge.feature:4
    Then I investigate      # features/hogehoge.feature:5
    When I know the meaning # features/hogehoge.feature:6

1 scenario (1 undefined)
3 steps (3 undefined)
0m0.066s

You can implement step definitions for undefined steps with these snippets:

Given(/^I want to explain scenario$/) do
  pending # Write code here that turns the phrase above into concrete actions
end

Then(/^I investigate$/) do
  pending # Write code here that turns the phrase above into concrete actions
end

When(/^I know the meaning$/) do
  pending # Write code here that turns the phrase above into concrete actions
end

\end{lstlisting}
ここではステップ定義に使用することができるコードブロックが表示されています.
ステップ定義はステップを作成するための方法です.このサンプルでは,Giver(), When(), Then()の
3つのメソッドを使ってステップを記述します.
これらのメソッドはそれぞれ\/\/で囲まれたRegexp(正規表現)とブロックを受け取ります.
Cucumberはシナリオの最初のステップを読み取り,そのステップにマッチする正規表現を持つステップ定義を探します.
その中の対応するステップ定義のブロックを実行します.

これはfeaturesディレクトリの下にstep\_definitionsディレクトリーにあることになっています.
このシナリオを成功させるには,Cucumberが読み込めるファイルにステップ定義を保存する必要があります[1, pp15.].
その内容は次の通りcucumberから自動生成されます.
\begin{lstlisting}[style=]

Given(/^I want to explain scenario$/) do
  pending # Write code here that turns the phrase above into concrete actio\
ns  
end

Then(/^I investigate$/) do
  pending # Write code here that turns the phrase above into concrete actio\
ns  
end

When(/^I know the meaning$/) do
  pending # Write code here that turns the phrase above into concrete actio\
ns  
end

\end{lstlisting}
pendingを削除して,そこにあれば良いなと思うコードを記述していきます.
ここまでがCucumberの使用方法のテンプレートです.



\subsection{my\_helpについて}
my\_helpは本研究室の西谷が開発したものです.
my\_helpとはユーザメモソフトであり,CUIスキルの習得を助けてくれます.
tarminal上で簡単に提示させることができるため,プログラミングに集中することができるといった特徴があります.
また,自分の見やすいように初心者でも簡単に編集することができ,すぐに参照できるメモとしても使うことができます.

\subsubsection{my\_helpのインストール}
githubに行ってdaddygongonのmy\_helpをforkします.

\begin{enumerate}
\item git clone git@github.com:daddygongon/my\_help.git
\item cd my\_help
\item rake to\_yml
\item rake clean\_exe
\item [sudo] bundle exec exe/my\_help -m
\item source ~/.zshrc or source ~/.cshrc
\item my\_help -l
\item rake add\_yml
\end{enumerate}
\subsubsection{my\_helpの更新}
git hubを用いてmy\_helpを新しくします.

\begin{enumerate}
\item git remote -vをする(remoteの確認).
\item (upstreamがなければ)git remote add upstream git@github.com:gitname/my\_help.git
\item git add -A
\item git commit -m 'hogehoge'
\item git push upstream master(ここで自分のmy\_helpをupstreamに送っとく)
\item git pull origin master(新しいmy\_helpを取ってくる)
\end{enumerate}
次にとってきた.ymlを~/.my\_helpにcpする.

\begin{enumerate}
\item cd my\_helpでmy\_helpに移動.
\item cp hogehoge.yml ~/.my\_help
\end{enumerate}
それを動かすために
(sudo)bundle exec ruby exe/my\_help -mをする.
ここで過去にsudoをした人はpermissionがrootになっているので,sudoをつけないとerrorが出ます.
(sudoで実行していたら権限がrootに移行される)

\begin{enumerate}
\item 新しいターミナルを開いて動くかチェックする.
\end{enumerate}
\subsubsection{my\_helpの特徴や問題点}
以下はmy\_helpのREADMEです[2].

CUI(CLI)ヘルプのUsage出力を真似て,user独自のhelpを作成・提供するgem.

\begin{enumerate}
\item 問題点
\end{enumerate}
CUIやshell, 何かのプログラミング言語などを習得しようとする初心者は,
commandや文法を覚えるのに苦労します.少しのkey(とっかかり)があると
思い出すんですが,うろ覚えでは間違えて路頭に迷います.問題点は,
- manは基本的に英語
- manualでは重たい
- いつもおなじことをwebで検索して
- 同じとこ見ている
- memoしても,どこへ置いたか忘れる

などです.

\begin{enumerate}
\item 特徴
\end{enumerate}
これらをgem環境として提供しようというのが,このgemの目的です.
仕様としては,
- userが自分にあったmanを作成
- 雛形を提供
\begin{quote}\begin{verbatim}
 - おなじformat, looks, 操作, 階層構造
\end{verbatim}\end{quote}
- すぐに手が届く
- それらを追加・修正・削除できる

hikiでやろうとしていることの半分くらいはこのあたりのことなの
かもしれません.memoソフトでは,検索が必要となりますが,my\_helpは
key(記憶のとっかかり)を提供することが目的です.



\section{結果}
CucumberとRSpecを用いてBDDでmy\_helpのテスト開発を進めて行きました.
ここでは,焦点を合わせたmy\_helpの中での一つの振る舞いである
「todoの更新」を例として詳しく見て行きます.

\subsection{todoの更新マニュアル}
最初に,todoを更新するときの手順を示します.

\begin{enumerate}
\item my\_todo --editを入力して~/.my\_help/my\_todo.ymlを開く
\item editorでtodoを書き込む(今週やることならweeklyというitemを作ってそこに書き込む)
\item 保存して~/.my\_help/my\_todo.ymlを閉じる
\end{enumerate}
この振る舞いがきちんとできているのかをBDDでテストを書いていきます.

\subsection{Cucumber}
以下はtodoの更新を行うときのfeatureです.
まず,適当なディレクトリにfeaturesというディレクトリを作成します.
次に先ほど作成した,featuresディレクトリにmy\_todo.featureを作成します.
\begin{lstlisting}[style=customCsh,basicstyle={\scriptsize\ttfamily}]
# language: ja 

機能: todoの更新を行う
自分のするべきことを書き込むためのtodoを更新する

 シナリオ: コマンドを入力してtodoを更新していく
          前提 todoを編集したい
          もし "my_todo --edit"と入力する
          ならば editが開かれる
          かつ 自分のtodoを書き込む
\end{lstlisting}
機能とは,このシステムの機能のことを記述します.ここでは,todoを更新するシステムですので,「todoの更新を行う」です.
機能の下には,機能の補足説明を記述します.機能の補足説明では,ルールがないので自分がわかりやすいように,記述するのが常識です.
シナリオは,その名の通りtodoを更新する時のユーザの行動やシステム振る舞いを前提,もし,ならば,かつ,しかしに分類して記述します.

ここまでfeatureが記述できたら,次はcucumberコマンドを実行してみます.
コマンドは以下の通りです.
 /Users/nasubi/nasu% cucumber features/my\_todo.feature 
featuresディレクトリにあるmy\_todo.featureファイルをcucumberで実行するという意味です.

実行すると以下のようになります.
\begin{lstlisting}[style=customCsh,basicstyle={\scriptsize\ttfamily}]

<省略>

1 scenarios (1 undefined)
4 steps (4 undefined)
0m0.080s

You can implement step definitions for undefined steps with these snippets:

前提(/^todoを編集したい$/) do
  pending # Write code here that turns the phrase above into concrete actions
end

もし(/^"([^"]*)"と入力する$/) do |arg1|
  pending # Write code here that turns the phrase above into concrete actions
end

ならば(/^editが開かれる$/) do
  pending # Write code here that turns the phrase above into concrete actions
end

ならば(/^自分のtodoを書き込む$/) do
  pending # Write code here that turns the phrase above into concrete actions
end
\end{lstlisting}
ここでは,2つのscenarioと7つのstepが失敗しています.
まだstep定義を記述していないので当たり前です.

一度cucumberを実行したのには理由があります.
featureを書いた時点でcucumberを実行すると,ステップ定義の元となるコマンドを,cucumberが自動的に作成してくれるからです.

以下がcucumberから出力されたステップ定義の元となる部分です.
\begin{lstlisting}[style=customCsh,basicstyle={\scriptsize\ttfamily}]
前提(/^todoを編集したい$/) do

end

もし(/^"([^"]*)"と入力する$/) do |command|

end

ならば(/^editが開かれる$/) do
  
end

ならば(/^自分のtodoを書き込む$/) do

end
\end{lstlisting}
これをコピーして,featuresディレクトリの中でstep\_definitionsディレクトリを作成し,その中にmy\_todo\_spec.rbを作成し,そこに貼付けます.

ここでもう一度cucumberを実行してみると
\begin{lstlisting}[style=customCsh,basicstyle={\scriptsize\ttfamily}]
/Users/nasubi/nasu% cucumber features/my_todo.feature 
# language: ja
機能: todoの更新を行う
自分のするべきことを書き込むためのtodoを更新する

  シナリオ: コマンドを入力してtodoを更新していく # features/my_todo.feature:6
    前提todoを編集したい             # features/step_definitions/my_todo_spec.rb:1
      TODO (Cucumber::Pending)
      ./features/step_definitions/my_todo_spec.rb:2:in `/^todoを編集したい$/'
      features/my_todo.feature:7:in `前提todoを編集したい'
    もし"my_todo --edit"と入力する  # features/step_definitions/my_todo_spec.rb:5
    ならばeditが開かれる             # features/step_definitions/my_todo_spec.rb:9
    かつ自分のtodoを書き込む           # features/step_definitions/my_todo_spec.rb:13

1 scenarios (1 pending)
4 steps (3 skipped, 1 pending)
0m0.045s

\end{lstlisting}
と変化が出てきます.
\begin{quote}\begin{verbatim}
1 scenarios (1 pending)
4 steps (3 skipped, 1 pending)
\end{verbatim}\end{quote}
これは1つのシナリオがあり,1つがpendingであり,4つのstepの内1つがpendingで3つがskipしたことを表しています.
step\_definitionsのmy\_todo\_spec.rbのpending部分を書き換えて,step\_definitionsの記述を進めて行きます.

まず,「前提」を見てみるとmy\_helpが何か振る舞いをすることはありません.
よって,このままにしておきます.
「もし」もユーザが入力するコマンドであり,my\_helpが何か振る舞いをすることはないのでこのままにしておきます.
次に,「ならば」を見てみるとmy\_helpがeditを開くという振る舞いをしています.
ステップ定義では,あれば良いと思うコードを記述するので私は下記のように記述しました.
\begin{lstlisting}[style=customCsh,basicstyle={\scriptsize\ttfamily}]
ならば(/^editが開かれる$/) do
  Mytodo::Edit.new.open
end
\end{lstlisting}
pendingの部分が書けたので,もう一度cucumber features/my\_todo.featureを実行します.
すると,下記のような結果が返ってきました.
\begin{lstlisting}[style=customCsh,basicstyle={\scriptsize\ttfamily}]
/Users/nasubi/my_help% cucumber features/my_todo.feature              
# language: ja
機能: todoの更新を行う
自分のするべきことを書き込むためのtodoを更新する

  シナリオ: コマンドを入力してtodoを更新していく # features/my_todo.feature:6
    前提todoを編集したい             # features/step_definitions/my_todo_spec.rb:20
    もし"my_todo --edit"と入力する  # features/step_definitions/my_todo_spec.rb:24
    ならばeditが開かれる             # features/step_definitions/my_todo_spec.rb:27
      uninitialized constant Mytodo (NameError)
      ./features/step_definitions/my_todo_spec.rb:28:in `/^editが開かれる$/'
      features/my_todo.feature:9:in `ならばeditが開かれる'
    かつ自分のtodoを書き込む           # features/step_definitions/my_todo_spec.rb:31

Failing Scenarios:
cucumber features/my_todo.feature:6 # シナリオ: コマンドを入力してtodoを更新していく

1 scenarios (1 failed)
4 steps (1 failed, 1 skipped, 2 passed)
0m0.055s

\end{lstlisting}
Cucumberは,エラーが出たステップのすぐ後ろにエラーを表示してくれます.
ここでCucumberでエラーが出たので,この「ならば editが開かれる」のシナリオに注目してRSpecに進むことにします.

\subsection{RSpec}
次にRSpecを使って実際にtodoを更新する振る舞いをするコード書いていきます.

そのための準備として,まずspecというディレクトリを作成し,my\_todoというサブディレクトリを追加します.
次に,このサブディレクトリにtodo\_spec.rbというファイルを追加します.
作業を進める過程で,lib/my\_todo/my\_todo.rbソースファイルとspec/my\_todo/todo\_spec.rbスペックファイルが1対1に対応するといった要領で,
並列のディレクトリ構造を築いていきます.
この機能はmy\_help --editと入力されれば,~/.my\_help/my\_todo.ymlが開かれるのでその振る舞いをするコードを書きます.
まずtodo\_spec.rbは下記の通りになります.
\begin{lstlisting}[style=customRuby,basicstyle={\scriptsize\ttfamily}]
require 'spec_helper'


module Mytodo
  describe Todo do
    describe "#open" do
      it "open file my_todo.yml" 
    end
  end
end

\end{lstlisting}
describe()メソッドは,RSpecのAPIにアクセスしてRSpec::Core::ExampleGroupのサブクラスを返します.
ExampleGroupクラスはオブジェクトに期待される振る舞いのサンプルを示すグループです.
it()メソッドはサンプルを作成します.

完成したコードを下記の通りです.
\begin{lstlisting}[style=customRuby,basicstyle={\scriptsize\ttfamily}]
require 'spec_helper'


module Mytodo
  describe Todo do
    describe "#open" do
      it "open file my_todo.yml" do
          system("emacs ~/.my_help/my_todo.yml")
      end
    end
  end
end

\end{lstlisting}
specディレクトリのmy\_todoディレクトリをrspecで実行すると下記のような結果がでました.
--colorを付け加えるとわかりやすく色づけをしてくれて,見やすくなります.
\begin{lstlisting}[style=customCsh,basicstyle={\scriptsize\ttfamily}]
/Users/nasubi/my_help% rspec spec/my_todo/ --color
/Users/nasubi/my_help/spec/my_todo/todo_spec.rb:6:in `<module:Mytodo>': uninitialized constant Mytodo::Edit (NameError)
	from /Users/nasubi/my_help/spec/my_todo/todo_spec.rb:5:in `<top (required)>'
	from /Users/nasubi/.rbenv/versions/2.2.2/lib/ruby/gems/2.2.0/gems/rspec-core-3.5.4/lib/rspec/core/configuration.rb:1435:in `load'
	from /Users/nasubi/.rbenv/versions/2.2.2/lib/ruby/gems/2.2.0/gems/rspec-core-3.5.4/lib/rspec/core/configuration.rb:1435:in `block in load_spec_files'
	from /Users/nasubi/.rbenv/versions/2.2.2/lib/ruby/gems/2.2.0/gems/rspec-core-3.5.4/lib/rspec/core/configuration.rb:1433:in `each'
	from /Users/nasubi/.rbenv/versions/2.2.2/lib/ruby/gems/2.2.0/gems/rspec-core-3.5.4/lib/rspec/core/configuration.rb:1433:in `load_spec_files'
	from /Users/nasubi/.rbenv/versions/2.2.2/lib/ruby/gems/2.2.0/gems/rspec-core-3.5.4/lib/rspec/core/runner.rb:100:in `setup'
	from /Users/nasubi/.rbenv/versions/2.2.2/lib/ruby/gems/2.2.0/gems/rspec-core-3.5.4/lib/rspec/core/runner.rb:86:in `run'
	from /Users/nasubi/.rbenv/versions/2.2.2/lib/ruby/gems/2.2.0/gems/rspec-core-3.5.4/lib/rspec/core/runner.rb:71:in `run'
	from /Users/nasubi/.rbenv/versions/2.2.2/lib/ruby/gems/2.2.0/gems/rspec-core-3.5.4/lib/rspec/core/runner.rb:45:in `invoke'
	from /Users/nasubi/.rbenv/versions/2.2.2/lib/ruby/gems/2.2.0/gems/rspec-core-3.5.4/exe/rspec:4:in `<top (required)>'
	from /Users/nasubi/.rbenv/versions/2.2.2/bin/rspec:23:in `load'
	from /Users/nasubi/.rbenv/versions/2.2.2/bin/rspec:23:in `<main>'

\end{lstlisting}
エラーが出てしまっているのがわかると思います.

 `<module:Mytodo>': uninitialized constant Mytodo::Edit (NameError)

上記のエラーを解決するために,specディレクトリの一つ上の構造のディレクトリにlibディレクトリを作成します.
その中にmy\_todoというディレクトリを作成し,my\_todo.rbを作成します.

構造を表示すると以下のようになっています.
 /Users/nasubi/my\_help/lib/my\_todo% ls
 my\_todo.rb

my\_todo.rbに先ほどのエラーでMytodoというmoduleがないといわれているので,
ここで作成します.

/Users/nasubi/my\_help/lib/my\_todo/my\_todo.rbの中に以下のコードを記述します.
\begin{lstlisting}[style=customCsh,basicstyle={\scriptsize\ttfamily}]
module Mytodo
  class Edit
    def open

    end
  end
end
\end{lstlisting}
また,これをrequireいないといけないので,lib/todo.rbとして,以下を追加します.
\begin{quote}\begin{verbatim}
require 'my_todo/my_todo'
\end{verbatim}\end{quote}
これだけでは,rspecがlib/my\_todo/my\_todo.rbを読み込んでいないため,このスペックを実行するために,specディレクトリにspec\_helper.rbに以下を追加します.
\begin{lstlisting}[style=customCsh,basicstyle={\scriptsize\ttfamily}]
$LOAD_PATH.unshift File.expand_path('../../lib', __FILE__)
require 'todo'
\end{lstlisting}\begin{lstlisting}[style=customRuby,basicstyle={\scriptsize\ttfamily}]
$LOAD_PATH.unshift File.expand_path('../../lib', __FILE__)
require 'todo'
\end{lstlisting}
これでもう一度rspecを実行してみます.
\begin{lstlisting}[style=customCsh,basicstyle={\scriptsize\ttfamily}]
/Users/nasubi/my_help% rspec spec/my_todo --color

Mytodo::Edit
  #open
    open file my_todo.yml

Finished in 2.97 seconds (files took 0.30255 seconds to load)
1 example, 0 failures

\end{lstlisting}
エラーが消えて成功しているのがわかります.
これで「ならば editが開かれる」のシナリオのRSpec部分が成功しました.
Red, Greenと進めたので次はRefactoringをするのですが,ここではあまり必要のないので省略します.
RSpecが終わったので,Cucumberに戻ります.

先ほどlibディレクトリでlib/my\_todo/my\_todo.rbを作成したので,cucumberでも読み込むために,以下を作成します.

 /Users/nasubi/my\_help/features/support/env.rb

env.rbの中は以下の通りです.
\begin{lstlisting}[style=customRuby,basicstyle={\scriptsize\ttfamily}]
$LOAD_PATH.unshift File.expand_path('../../../lib', __FILE__)
require 'todo'
\end{lstlisting}
これでcucumberもlib/の中身を読み取ってくれます.

もう一度cucumberを実行してみると,
\begin{lstlisting}[style=customCsh,basicstyle={\scriptsize\ttfamily}]
/Users/nasubi/my_help% cucumber features/my_todo.feature 
# language: ja
機能: todoの更新を行う
自分のするべきことを書き込むためのtodoを更新する

  シナリオ: コマンドを入力してtodoを更新していく # features/my_todo.feature:6
    前提todoを編集したい             # features/step_definitions/my_todo_spec.rb:2
    もし"my_todo --edit"と入力する  # features/step_definitions/my_todo_spec.rb:6
    ならばeditが開かれる             # features/step_definitions/my_todo_spec.rb:9
    かつ自分のtodoを書き込む           # features/step_definitions/my_todo_spec.rb:13

1 scenarios (1 passed)
4 steps (4 passed)
0m0.027s
\end{lstlisting}
エラーが消えています.

これでBDDが成功しました.
残りの「自分のtodoを書き込む」もmy\_helpが何か振る舞いをするわけではないので,「コマンドを入力してtodoを更新する」シナリオ全てのテスト開発が終わりました.
このようにBDDでmy\_helpのテスト開発を行っていきました.



\subsubsection{featuresでの記述とその意味}
featuresでの記述は,コマンドの振る舞いを説明する自然な記述となる.
その様子をspecific\_helpが用意しているデフォルトのコマンドについて
説明する.specific\_helpとは,ユーザが作成するそれぞれのヘルプである.
speific\_helpの--helpを表示させると,
\begin{quote}\begin{verbatim}
        --edit                       edit help contentsを開く
        --to_hiki                    hikiのformatに変更する
        --all                        すべてのhelp画面を表示させる
        --store [item]               store [item] でback upをとる
        --remove [item]              remove [item] back upしてるlistを消去する
        --add [item]                 add new [item]で新しいhelpを作る
        --backup_list [val]          back upしているlistを表示させる
\end{verbatim}\end{quote}
が得られる.これらの項目について順に詳細な振る舞いとそれを記述する
シナリオを検討していく.

\subsubsection{my\_helpのfeatures}
下記は私が作成したmy\_helpの一部をfeaturesで書いたものである.

\paragraph{--add [item]}
このコマンドは新しいitemをspecific\_helpに追加する.
提供される機能を
シナリオの先頭に内容をかいつまんでこの振る舞いが記述されている.
実装では,ヘルプの内容は~/.my\_help/emacs\_help.ymlに元dataがある
\begin{quote}\begin{verbatim}
nasu% cat add.feature
#language: ja

#--add [item]
機能: 新しいitemをspecific_helpに追加する
specific_helpとは,ユーザが作成するそれぞれのヘルプである
新しいhelp画面を追加したい

シナリオ: コマンドを入力してspecific_helpにitemを追加する
        前提 新たなhelpコマンドを追加したい
        もし emacs_help --add[item]を入力する
        ならば ~/.my_help/emacs_help.ymlに新しいitemが自動的に追加される

\end{verbatim}\end{quote}
\paragraph{全てのhelp画面の表示}\begin{lstlisting}[style=,basicstyle={\scriptsize\ttfamily}]

#language: ja

#--all
機能: 全てのhelp画面を見る
複数のhelp画面を一度に見たい時に便利である

シナリオ: コマンドを入力してすべてのhelpを見る
        前提 複数のhelp画面を表示したい
        もし emacs_help --allと入力する
        ならば すべてのhelp画面が表示される
\end{lstlisting}
シナリオ:コマンドをニュ力してすべてのhelp画面を見る

コマンド:emacs\_help --all

\paragraph{過去にバックアップしてあるitemのリストの表示}\begin{lstlisting}[style=,basicstyle={\scriptsize\ttfamily}]
#language: ja

#--backup_list
機能: 過去にバックアップしてあるitemのリストを表示させる
何をバックアップしたかの確認をしたい

シナリオ: コマンドを入力してバックアップのリストを見る
        前提 バックアップのリストを見たい
        もし emacs_help --backup_listを入力する
        ならば バックアップしているitemのリストが表示される
        
\end{lstlisting}
シナリオ:コマンドを入力してバックアップのリストを見る
コマンド:emacs\_help --backup\_list

\paragraph{helpコマンドの追加や削除,編集をするファイルの開示}\begin{lstlisting}[style=,basicstyle={\scriptsize\ttfamily}]
# language: ja
#--edit
機能: helpコマンドの追加や削除,編集をするためのeiditを開く
emacs_helpと入力したときに出てくるhelpのコマンドの追加や削除,編集ができる

シナリオ: コマンドを入力してeditを開く
        前提 emacs_helpのコマンドの編集がしたい
        もし emacs_help --editと入力する
        ならば ~/.my_help/emacs_help.ymlがemacsで開かれる
\end{lstlisting}
シナリオ:コマンドを入力してeditを開く
コマンド:emacs\_help --edit

元dataである~/.my\_help/emacs\_help.ymlを開く.

ここで編集を行い,emacsで開いているのでC-x,C-sで保存する.

\paragraph{specific\_helpのitemの消去}\begin{lstlisting}[style=,basicstyle={\scriptsize\ttfamily}]
#language: ja

#--remove [item]
機能: specific_helpのitemを消す
いらなくなったitemを消したいときに使う

シナリオ: コマンドを入力してitemを消す
        前提 いらないitemを消したい
        もし emacs_help remove [item]
        ならば ~/.my_help/emacs_help.ymlからitemが消える

\end{lstlisting}
シナリオ:コマンドを入力してitemを消す

コマンド:emacs\_help --remove

\paragraph{itemのバックアップ}\begin{lstlisting}[style=,basicstyle={\scriptsize\ttfamily}]
#language: ja

#--store [item]
機能: itemのバックアップを取る
バックアップとして残したいitemがあるときに使う

シナリオ: コマンドを入力してitemのバックアップをとる
        前提 バックアップをとっておきたい
        もし emacs_help --store [item]と入力する
        ならば 入力したitemのバックアップが作られる
\end{lstlisting}
シナリオ:コマンドを入力してバックアップをとる

コマンド:emacs\_help --store [item]

\paragraph{hikiへのformatの変更}\begin{lstlisting}[style=,basicstyle={\scriptsize\ttfamily}]
# language: ja

#--to_hiki
機能:formatをhikiモードに変更する
一つ一つエディタで開いて変更するのがめんどくさい時に有益である

シナリオ: コマンドを入力してformatをhikiモードに変える
        前提 hikiモードに変更したい
        もし emacs_help --to_hikiと入力する
        ならば formatがhikiモードに変更される
\end{lstlisting}
シナリオ:コマンドを入力してformatをhikiモードにする

コマンド:emacs\_help --to\_hiki

\paragraph{todoの更新}\begin{lstlisting}[style=,basicstyle={\scriptsize\ttfamily}]
# language: ja

機能: todoの更新を行う
todoは更新していくものであり,新しく書いたり終わったものを消したいのでバック\
アップをとって,過去のtodoを残しておく

シナリオ: コマンドを入力してtodoを更新していく
          前提 todoを編集したい
          もし "my_todo --edit"と入力する
          ならば editが開かれる
          かつ 自分のtodoを書き込む

シナリオ: コマンドを入力してバックアップをとる
          前提 todoの編集が終わった
          もし "my_todo --store [item]"と入力する
          ならば itemのバックアップを取る
\end{lstlisting}
シナリオ1:コマンドを入力してtodoを更新する
シナリオ2:コマンドを入力してバックアップをとる

コマンド1:my\_todo --edit
コマンド2:my\_todo --store [item]

my\_todo --editで~/.my\_help/my\_todo.ymlを開く.

ここで編集を行い,emacsで開いているのでC-x,C-sで保存する.

my\_todo --store [item]でtodoのitemをバックアップとっておく.

この動作により過去のバックアップを閲覧することができ,どんどん更新することが可能である.


\section{考察}
 今回の開発において,ユーザメモソフトであるmy\_helpのテスト開発は成功した.
つまり,my\_helpの仕様と動作の標準が確定したことで,今後のソフトを進化させるための共同開発がスムーズにいくと推測される.
先に述べた通り,ソフト開発は一人でせず,複数人で開発することが普通である.
その時に起こる障害として,意思の疎通ができていないことがあげられる.
振る舞いが標準化されていないと,どのような振る舞いをするのかが,プログラムを見るだけでは,ずれが生じてしまう.
また,開発者の意図が読めていないと,コードの意味も変わってくるので,テスト開発をして,仕様と動作の標準を確定することは重要であった.
加えて,初めてmy\_helpを使うユーザでも仕様と動作の標準がわかっていれば,短い時間でmy\_helpを使いこなせるし,そのことによって,ユーザそれぞれに自分にあったmy\_helpに変更することも容易になる.


\section{謝辞}
 本研究を進めるにあたり,様々なご指導を頂きました西谷滋人教授先生に深く感謝いたします.
また,本研究の進行に伴い,様々な助力,知識の供給を頂きました西谷研究室の同輩,先輩方に心から感謝の意を示します.
本当にありがとうございました.

\section{参考文献}
\begin{enumerate}
\item The RSpec Book 著者:David Chelimsky Dave Astels Zach Dennis ほか 翻訳:株式会社クイーブ 監修:株式会社クイーブ 角谷信太郎 豊田裕司.
\item Shigeot R. Nishitani, my\_helpのREADME, \verb|http://www.rubydoc.info/gems/my_help/0.4.3|.
\end{enumerate}
\end{document}
